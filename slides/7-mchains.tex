\documentclass[aspectratio=169]{beamer}
\usepackage{etex} % fixes new-dimension error
\usepackage{lmodern}
\usepackage[T1]{fontenc}

\usepackage{graphicx,amsmath}
\usepackage{stmaryrd} % cf. interleave
\usepackage{booktabs}
\usepackage{amscd}
\usepackage{multicol}
\usepackage[absolute,overlay]{textpos}
\usepackage{alltt}
\usepackage{proof}
\usepackage{subcaption}
\usepackage{tikz}
\usepackage{tikz-cd}
\usepackage[new]{old-arrows}
\usepackage[all]{xy}
\usepackage{pgfplots}
\usepackage{textcomp}

\input{macros/macros}
%-------------- template --------------------------------------------------
\input{macros/beamerconf}
%----------------------------------------------------------------------------

%------ using pstricks (rnode etc) ------------------------------------------
% \usepackage{pstricks,pst-node,pst-text,pst-3d}

% % context
% \AtBeginSection[]
% {
%     \begin{frame}
%         \frametitle{Table of Contents}
%         \tableofcontents[currentsection]
%     \end{frame}
% }
% \author[Renato Neves]{Renato Neves}
% % logos of institutions
% \titlegraphic{
%   \begin{textblock*}{5cm}(6.7cm,7.45cm)
%      \includegraphics[scale=0.06]{images/uminho.png}\hspace*{.85cm}~%
%   \end{textblock*}
%   \begin{textblock*}{5cm}(9.4cm,7.45cm)
%     \includegraphics[scale=0.50]{images/haslab.pdf}
%   \end{textblock*}
% }
% % No date
% \date{}


\begin{document}


% \title{
%   Time-critical reactive systems \\ (Verification )  }

% \frame[plain]{\titlepage}

\setLecture{6}{Probabilities: Markov chains and statistical model checking}

\frame[plain]{\titlepage}

\section*{Where we are}

\begin{slide}{Syllabus}
  \frsplitt{
    \begin{itemize}
      \item Introduction to model-checking
      \item CCS: a simple language for concurrency
      \begin{itemize}
        \item Syntax
        \item Semantics
        \item Equivalence
        \item {mCRL2}: modelling
      \end{itemize}
      \item Dynamic logic
      \begin{itemize}
        \item Syntax
        \item Semantics
        \item Relation with equivalence
        \item {mCRL2}: verification
      \end{itemize}
    \end{itemize}
  }{
    \begin{itemize}
      \item  {Timed Automata}
      \begin{itemize}
        \item Syntax
        \item Semantics (composition, Zeno)
        \item Equivalence
        \item {UPPAAL}: modelling
      \end{itemize}
      \item Temporal logics (LTL/CTL)
      \begin{itemize}
        \item Syntax
        \item Semantics
        \item {UPPAAL}: verification
      \end{itemize}
      \alert{
      \item Probabilistic and stochastic systems
      \begin{itemize}
        \item Going probabilistic
        \item \structure{UPPAAL}: monte-carlo
      \end{itemize}
      }
    \end{itemize}
  }
\end{slide}

\section{Going probabilistic}

%----------------------------------------------------------------------------------
\begin{frame}[t]\frametitle{Motivation}
  \begin{block}{Systems can get very complex}
    \begin{itemize}
      \item E.g., 5 components, 3 possible traces each
      \item No communication (pure interleaving)
      \item 5 traces (or 7 components) -- becomes X
      \\[5mm]
      \pause
      \item Verifying deadlock freedom (and others) requires traversing all states
      \item \structure{Approximation}:
      \begin{itemize}
        \item traverse only part of the states
        \item give more \structure{priority} to some actions
        \item return (statistically) likelihood of a given property
      \end{itemize}
    \end{itemize}
  \end{block}
\end{frame}

\begin{frame}[t]\frametitle{Recall: A taxonomy of transition systems}
  \begin{itemize}
    \item $\alpha: S \to \Act \times S$ \alert{Moore machine}
    \item $\alpha: S \to \mathtt{Bool} \times S^{\Act}$ \alert{deterministic automata}
    \item $\alpha: S \to \mathtt{Bool} \times \tikzmark{a1} \Pow(S)^{\Act} \tikzmark{a2}$ \alert{non-deterministic automata (reactive)}
    \\~
    \item $\alpha : S \to \Pow( N \times S )$ \alert{non deterministic LTS (generative)}
    \item $\alpha : S \to (S + 1)^N$ \alert{partial deterministic LTS} % (reactive)}
    \item $\alpha : S \to \Pow(S)$ \alert{unlabelled TS}
    % \item $\alpha : S \to N \times S + 1$ \alert{partial LTS (generative)}
    % \item $\alpha : S \to P(S)^N$ \alert{non deterministic LTS (reactive)}
    \\~
    \item $\alpha: S \to \tikzmark{b1} \Dist(S) \tikzmark{b2}$ \alert{Markov chain}

  \end{itemize}  


\end{frame}

\begin{frame}[t]\frametitle{Bringing probabilities to transition systems}
  
  \begin{alertblock}{Markov chains}
    $$\alpha : S \to \Dist(S)$$
    where $D(S)$ is the set of all \alert{discrete probability distributions} on set $S$

    A Markov chain goes from a state $s$ to a state $s'$ with probability $p$ if
    $$\alpha(s) = \mu\text{~~~with~~~} \mu(s') = p > 0$$
  \end{alertblock}

  % \begin{block}{}
  %   \begin{itemize}
  %     \item $\mu: \Dist(S)$
  %     \item $0\leq \mu(s)\leq 1$
  %     \item $\sum_{s\in S} \mu(s) = 1$
  %     \item 
  %   \end{itemize}
  % \end{block}

% Notation
% s->mu  and s-p->s' -- squigly
\end{frame}


\begin{frame}[t]\frametitle{Recall discrete distributions}

  \begin{block}{Recall}
  $\mu: S\to [0,1]$ is a \alert{discrete probability distribution} if
  \begin{itemize}
  \item $\{s \in S ~|~ \mu(s) >0\}$, is finite (called the \structure{support} of $\mu$), and
  \item $\sum_{s\in S} \mu(s) = 1$
  \end{itemize}
  \end{block}

  \begin{exampleblock}{Examples}
  \begin{itemize}
        \item \alert{Dirac distribution}: $\mu^1_s = \{s \to 1\}$
        \item \alert{Product distribution}: $(\mu_1 \times\mu_2)\pair{s,t} = \mu_1(s) \times \mu 2(t)$
  \end{itemize}    
  \end{exampleblock}
\end{frame}


\begin{frame}[t]\frametitle{Example}
  \centering

    \wrap{\begin{tikzpicture}[aut]
      \node[st,initial above,final]  (s_0) {$A_0$};
      \node[st] (s_1) [above right of=s_0] {$A_1$};
      \node[st] (s_2) [below right of=s_0] {$A_2$};
      \path[->]
        (s_0) edge  node {0.4}  (s_1)
        (s_0) edge  node {0.4}  (s_2)
        (s_1) edge  node {0.7}  (s_2)
        (s_0) edge[loop left] node {0.2}  ()
        (s_1) edge[loop right] node {0.3}  (s_1)
        (s_2) edge[loop right] node {1}  (s_1);
    \end{tikzpicture}}
\end{frame}

\begin{frame}[t]\frametitle{Reactive PTS}
  \centering
  {\Large \alert{$\alpha: S \to (\Dist(S) + 1)^N$}}

  \bigskip

  \doExercise{Formalise the systems below as functions}{
  \centering
  \wrap{\begin{tikzpicture}[aut]
    \node[st,initial above,final]  (s_0) {$A_0$};
    \node[st] (s_1) [above right of=s_0] {$A_1$};
    \node[st] (s_2) [below right of=s_0] {$A_2$};
    \path[->]
      (s_0) edge  node {0.4}  (s_1)
      (s_0) edge  node {0.4}  (s_2)
      (s_1) edge  node {0.7}  (s_2)
      (s_0) edge[loop left] node {0.2}  ()
      (s_1) edge[loop right] node {0.3}  (s_1)
      (s_2) edge[loop right] node {1}  (s_1);
  \end{tikzpicture}}
  ~~~~~~~~~~~
    \wrap{\begin{tikzpicture}[aut]
    \node[st,initial above,final]  (s_0) {$A_0$};
    \node[st] (s_1) [above right of=s_0] {$A_1$};
    \node[st] (s_2) [below right of=s_0] {$A_2$};
    \path[->]
      (s_0) edge  node {a[0.6]}  (s_1)
      (s_0) edge  node {b[1]}  (s_2)
      (s_1) edge  node {b[0.7]}  (s_2)
      (s_0) edge[loop left] node {a[0.4]}  ()
      (s_1) edge[loop right] node {b[0.3]}  (s_1);
      % (s_2) edge[loop right] node {1}  (s_1);
  \end{tikzpicture}}

  }
  
  \begin{block}{}
    Notions of bisimulation arise naturally.
  \end{block}
\end{frame}





\section{Probabilities in Uppaal}

\begin{frame}[t]\frametitle{Stochastic Timed Automata -- examples}
  \centering    
  \wrap{\uppbox[60mm]{A1}{images/A1.pdf}}~~~~~~~
  \wrap{\uppbox[70mm]{A2}{images/A2.pdf}}\\[1mm]
  \wrap{\uppbox[50mm]{A3}{images/A3.pdf}}
\end{frame}

\begin{frame}[t]\frametitle{A1: When does it end?}
  \centering
  \splittwo{0.4}{0.58}{
    \wrap{\uppbox[60mm]{A1}{images/A1.pdf}}\\
    \wrap{\uppbox[60mm]{A1's histogram}{images/A1-hist.pdf}}
  }{
    \begin{itemize}
      \item Run 102000 times
      \item Histogram: how many times it took [9..9.1] seconds?
      \item ...
    \end{itemize}
  }
\end{frame}

\begin{frame}[t]\frametitle{A2: When does it end?}
  \centering
  \splittwo{0.4}{0.58}{
    \wrap{\uppbox[50mm]{A2}{images/A2.pdf}}\\
    \wrap{\uppbox[50mm]{A2's histogram}{images/A2-hist.pdf}}
  }{
    \begin{itemize}
      \item Run 100000 times
      \item Histogram: how many times it took [9..9.1] seconds?
      \item ...
    \end{itemize}
  }
\end{frame}

\begin{frame}[t]\frametitle{A3: When does it end?}
  \centering
  \splittwo{0.4}{0.58}{
    \centering
    \wrap{\uppbox[30mm]{A3}{images/A3.pdf}}\\
    \wrap{\uppbox[60mm]{A3's histogram}{images/A3-hist.pdf}}

  }{
    \begin{itemize}
      \item Run 300000 times
      \item Histogram: how many times it took [9..9.1] seconds?
      \item ...
    \end{itemize}
  }
\end{frame}


\begin{frame}[t]\frametitle{Stochastic Automata}
  \centering
  {\Large \alert{$\alpha: S \to (\Dist(S\times Act \times ...) + 1) \times (R + \cc{C})$}}

  \bigskip
  ...
\end{frame}



\section{Probabilistic queries in Uppaal}

...


\end{document}
