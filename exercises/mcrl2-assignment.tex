\documentclass[11pt]{article}


\input{macros}

% \renewcommand{\bash}[1]{\texttt{#1}\xspace}
\date{System Verification -- 2024/2025}

% \usepackage[latin1]{inputenc} % pt characters
\usepackage[utf8x]{inputenc} % pt characters and eur sign (instead of latin1)

% \usepackage{draftwatermark}
% \SetWatermarkText{Draft}
% \SetWatermarkScale{1.4}

\begin{document}
 
% --------------------------------------------------------------
%                         Start here
% --------------------------------------------------------------
 
\title{Modelling and verifying behaviour in mCRL2}

\author{Jos\'{e} Proen\c{c}a} 


\maketitle

% \vspace*{-2mm}
\descrbox{To do}{
  Produce a report as a PDF document including the answers to the exercises below.
  Read, modify, and produce mCRL2 specifications, following the instructions in the exercises.
  
  Auxiliary files in \textcolor{blue}{\url{https://fm-dcc.github.io/sv2425/exercises/farmer.zip}}
}

\descrbox{What to submit}{
  The PDF report and the files requested in the exercises: \bash{farmer1.mcrl2}, \bash{farmer2.mcrl2}, \bash{farmer3.mcrl}, and \bash{vending.mcrl2}. 
}

\descrbox{How to submit using git}{
  \begin{enumerate}[itemsep=1pt,leftmargin=20pt]
    \item Create a private git repository in your favouring host (e.g., github or bitbucket).
    \item \textbf{Name it \bash{SV2425-g<group number>}.}
    \item Add \bash{jose.proenca@fc.up.pt} as a member to the group (read-permissions are enough).
    \item Include all the files to be submitted in the repository.
  \end{enumerate}
  Note that \textbf{all students should push commits.}
}

\descrbox{Deadline}{
  4 Nov 2024 @ 23:59 (Monday) %6 Dec 2021 @ 23:59 (Monday)
}



\vspace{3mm}

\section*{The farmer-fox-goose-beans problem}

A \textbf{farmer} wants to transport a \textbf{fox}, a \textbf{goose}, and some \textbf{beans} across a river (from the \textbf{left} margin to the \textbf{right} margin).
Unfortunately, he can only carry one at a time. Furthermore, if the farmer is not present, the fox will eat the goose and the goose will eat the beans. The problem is solved if the farmer can carry all animals across the river.

\pagebreak

% \vspace*{-3mm}
\lstinputlisting[style=mcrl2]{farmer1.mcrl2}

\begin{myExercise} \label{ex:ba1}
We will encode the same problem using mCRL2's process algebra.
Start by downloading the auxiliary files for this assignment at \textcolor{blue!85}{\url{https://fm-dcc.github.io/sv2425/exercises/farmer.zip}}, where you will find the \bash{farmer1.mcrl2} file above.
This is a simplified (but not fully accurate) specification of our farmer-fox-goose-beans problem.


The specification is split into three sections:
\code{act}, a declaration of 24 actions,
\code{proc}, the definition of 4 processes, and
\code{init}, the initialisation of the system.


\subex{
Create a new project \bash{farmer1} using \bash{mcrl2ide}, and add the resulting project folder to your git repository.
Produce the labelled transition system (LTS) of this mCRL2 specification and
\textbf{show a screenshot of the LTS (make sure it is understandable).}
}

\subex{
This specification is not fully accurate yet, i.e., it does model all aspects of the puzzle.
\textbf{Explain informally why our model is not accurate}, by explaining what is being modelled and what is still missing.
}

\subex{If you replace the \code{init} block by only \code{Fox || Goose || Beans || Farmer} (i.e., without the restrictions \code{allow} and \code{comm}) \textbf{would you obtain more or less states} than with the original specification? \textbf{Why?}
}
\end{myExercise}

% ~\\[-6mm]

%----------------------------------------------
\begin{myExercise} \label{ex:ba2}
We present below a new specification for the same problem consisting of a single process \code{State} that keeps the state information, found in the provided auxiliary file \bash{farmer2.mcrl2}.
This new specification includes more advanced features of mCRL2, including:
a \emph{data structure}, actions with \emph{data parameters}, processes with \emph{data parameters}, and user defined \emph{functions} \code[morekeywords={inv}]{inv} and \code[morekeywords={ok}]{ok}.

% \clearpage


\lstinputlisting[style=mcrl2]{farmer2.mcrl2}


\subex{This new specification has a hole in the definition of \code[morekeywords={ok}]{ok}, marked with \textcolor{gray}{\code{\%\% (1) \%\%}}.
Extend the given mCRL2 definition by replacing this hole with the code that describes the desired state invariant and save the resulting specification in a new project named \bash{farmer2}.
\textbf{Show your new definition of the function \code[morekeywords={ok}]{ok}.}
}

\subex{\label{ex:ba3}
Without modifying the process \code{State}, adapt the specification by adding a new process \code{Counter(n:Nat)} that runs in parallel with \code{State(left,left,left,left)} and counts the number of traversals made by the boat. Save the resulting specification in a new project \bash{farmer3} and \textbf{show your new specification}. (Hint: it could be useful to use a bound for the \code{Counter}, i.e., do not allow $n$ to be bigger than a certain number.)
}
\end{myExercise}



% \begin{myExercise}
% You will now create variations of the \code{Sys} processes in the \code{proc} section of both \bash{farmer1.mcrl2} and \bash{farmer2.mcrl2}, as explained below, and compare them to the originals.

% % Recall the action-based \bash{farmer1} specification from Exercise~\ref{ex:ba1} and the state-based \bash{farmer2} specification from Exercise~\ref{ex:ba2}.


% \subex{Create a new process \code{SysHide} in both \bash{farmer1.mcrl2} and \bash{farmer2.mcrl2} that is equal to \code{Sys} except that it will also {hide all allowed actions except \code{win}} (using \code{hide}).
% \textbf{Show the LTS of both of the resulting \code{SysHide} processes (for each file).}
% }

% \subex{
% Create a new file \bash{farmer12.mcrl2} that combines both processes from \bash{farmer1.mcrl2} and \bash{farmer2.mcrl2}. We will now compare based on the notion of strong bisimulation.
% %
% For that do the following:
% \begin{itemize}
%   \item rename the process names \code{Sys} and \code{SysHide} that were imported from the two files to avoid having the same name. E.g., \code{Sys1} instead of \code{Sys} for the process imported from \bash{farmer1.mcrl2}.
%   \item Redefine (for now) the function \code[morekeywords={ok}]{ok} by setting it to true, i.e., define \code[morekeywords={ok}]{ok(fm,f,g,b)=true;}.
%   \item \textbf{Show the LTS of both processes \code{SysHide1} and \code{SysHide2}.}
%   \item \textbf{Compare them using strong bisimulation} by adding a new \textbf{Equivalence} property that compares them. %Note that you will have to delete the \code{init} block.
%  \end{itemize}
% \textbf{What can you conclude?}}

% % \subex{
% % Visualise the minimised LTS for \code{SysHide2} with respect to branching bisimulation, using the tool in \bash{Tools>Show reduced State Space}.
% % \textbf{Include a screenshot of the minimised LTS and explain what information can we infer from this LTS.}}
% \end{myExercise}

\begin{myExercise}
You will now verify properties of these systems.
In \bash{mcrl2ide}, a property can be written using \bash{Tools>Add Property}.
Answer the questions below on the use of mu-calculus for specifying properties in mCRL2.

\subex{\label{ex:ver1}
What does the property ``\code{[true*]<win>true}'' mean? Does it hold for \bash{farmer1} and for \bash{farmer2}?}

\subex{\label{ex:ver2}
Does the property ``\code{[true*.foxr.win]false}'' holds for \bash{farmer1}? Does the equivalent property ``\code{[true*.fox(right).win]false}'' holds for \bash{farmer2}? What can you conclude?}

\subex{\label{ex:ver3}
Recall that \bash{farmer1} is less complete than \bash{farmer2}, because it fails to include some important invariants.
Write a \textbf{single property} for \bash{farmer2} to capture that:
\begin{itemize}
  \item no bad state can be reached, and
  \item the goal can be reached (everyone can cross).
\end{itemize}
\textbf{Add this property} to your project and \textbf{verify} it using mCRL2 toolset.
\textbf{Reformulate} this property for \bash{farmer1} and add it to that project, and \textbf{verify} if it holds.

}

\subex{
Consider now the extended system \bash{farmer3} from Exercise~\ref{ex:ba3}.
In this example there is a an extra process called \code{Counter(n:Nat)}.
\textbf{Define the following two properties} over actions of this counter:
\begin{enumerate}
\item It is possible to win after exactly 7 moves.
\item It is not possible to win in less than 7 moves.
\end{enumerate}
}
\end{myExercise}


~\\

\section*{A vending machine}

\begin{myExercise} \label{ex:vm}
Specify two interacting processes in mCRL2:
\begin{itemize}
  \item \textbf{a vending machine} with 2 products, apples and bananas, costing 1eur and 2eur respectively; and
  \item \textbf{a user} who can insert 1eur or 2eur coins and request for products.
\end{itemize}

Provide a specification of this system and include them in a \bash{vending.mcrl2} file, according to the requirements below. Try to keep the specifications simple. \textbf{Submit this file in your git repository.}

Requirements:
\begin{itemize}
  \item The user must be able to get apples and bananas;
  \item The machine accepts up to 3eur, and not more than that;
  \item The machine must give change back when applicable;
  \item The machine can be powered off and powered on;
\end{itemize}

\end{myExercise}


\begin{myExercise} \label{ex:vm}
You will now specify two interacting processes in mCRL2:
\begin{itemize}
  \item \textbf{a vending machine} with 2 products, apples and bananas, costing 1€ and 2€ respectively; and
  \item \textbf{a user} who can insert 1€ or 2€ coins and request for products.
\end{itemize}

\subex{
Specify this system in a new file \bash{vending.mcrl2}
such that the properties below hold. Try to keep the specifications simple. 
}
\begin{lstlisting}
[true*.pay2eur.pay2eur] false
[true*.pay2eur.pay1eur] false
[true*.pay2eur]<(!pay1eur && !pay2eur)*.getApple> true
<true*.pay2eur.true*.getBanana> true \end{lstlisting}
\textbf{Submit this file in your git repository.}
\textbf{Show your specification} and \textbf{show a screenshot of its LTS}.

\vspace{2mm}
\subex{Now it is your turn to formalise the requirements. Write a set of requirements using mCRL2's formulas that capture the informal requirements stated below. Do not implement the mCRL2 model. If necessary, include an explanation of the actions and assumptions that you used.}
\begin{quote}\it
  \begin{enumerate}
    \item I would like the vending machine to sell 3 items: apples, bananas, and chocolates.
    \item It should be possible to buy chocolates for 2€ and fruit for 1€.
    \item Only 1€ and 2€ coins are accepted.
    \item The machine has a maximum capacity for 1€ coins and for 2€ coins. 
    % \item The machine does not accept coins if its capacity is full.
 %    \item The machine should give change back when buying fruit after inserting 2€.
 %    \item If the machine has already 2€ inserted, it refuses another coin.
    \item When the machine has no 1€ coins, it cannot not sell fruit with a 2€ coin. 
 %    \item The user can request the money back after inserting coins.
 \end{enumerate}
\end{quote}
\end{myExercise}





% % --------------------------------------------------------------
% \section*{LTS Equivalence}


% \begin{myExercise}  
% Recall the action-based \bash{farmer1.mcrl2} specification from Exercise~\ref{ex:ba1} and the state-based \bash{farmer2.mcrl2} specification from Exercise~\ref{ex:ba2}.


% \subex{Modify the initial process (\code{init}) of both \bash{farmer1.mcrl2} and \bash{farmer2.mcrl2} to {hide all allowed actions except \code{win}} (using \code{hide}), and save them as \bash{farmer1-tau.mcrl2} and \bash{farmer2-tau.mcrl2}, respectively.
% In \bash{farmer2-taus.mcrl2} {redefine the function \code[morekeywords={ok}]{ok}} by setting it to true, i.e., define \code[morekeywords={ok}]{ok(fm,f,g,b)=true;}.
% \textbf{Show the resulting \code{init} block from each file.}
% }

% \subex{
% Generate the \bash{.lts} files corresponding to \bash{farmer1-tau.mcrl2} and \bash{farmer2-tau.mcrl2}, and compare them using strong bisimulation using the following command.
% \textbf{What can you conclude?}}

% \begin{lstlisting}[style=bash]
% $\texttt{\$}$ ltscompare --equivalence=bisim farmer1-taus.lts farmer2-taus.lts
% \end{lstlisting}


% \subex{
% Using \bash{ltsconvert}, minimise the LTS for \bash{farmer2-taus.mcrl2} with respect to branching bisimulation, using the command below.
% \textbf{Include a screenshot of the minimised LTS and explain what information can we infer from this LTS.}}

% \begin{lstlisting}[style=bash]
% $\texttt{\$}$ ltsconvert --equivalence=branching-bisim farmer2-taus.lts
% \end{lstlisting}


% \end{myExercise}



 
\end{document}